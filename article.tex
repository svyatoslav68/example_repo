В этой статье будет рассмотрена возможность использования git для совместной работы над 
некоторым проектом. Понятнее будет рассматривать эту работу на некотором вымышленном примере. 
Пример учебный и абсолютно вымышленный. Все аналогии и совпадении совершенно случайны.

Предполагается, что читатель уже знаком с основами работы с git. Здесь мы будем рассматривать,
в основном, работу в команде с помощью git.
Итак:
Действующие лица:
Антон - человек очень занятый, но добровольно взявший на себя повинность модерации проекта.
Кирилл - человек безусловно занятый, но заинтересованный в успешности предприятия и добровольно
взявший на себя обязанности по личному участию в проекте.
Святослав - человек малозанятый и взявший на себя основное бремя проекта.

Действие первое. 
Служебное помещение, где проходит очное совещание, в котором принимают участие все действующие лица.

Антон: Коллеги, есть интересная задача. Мы все так или иначе пользуемся нагревательными приборами
для удовлетворения в том числе и духовных потребностей своего организма. Но электронагревательный 
прибор, как известно является источником множества опасностей. Было бы хорошо разработать
документацию, необходимую для использования нашего чайника. Какие будут соображения?

Кирилл: Да все, в общем, известно. Нужно написать инструкцию и перечень используемых приборов, 
утвердить их у соотвествующего отвественного работника и разместить все это рядом с используемым
электронагревательным прибором. Затем все этим пользоваться. 

Святослав: Как я понимаю, предлагается мне возглавить эту работу? Я к этому готов. В скорости
пришлю на согласование проект предлагаемых документов. Думаю, будет правильно и эффективно
для совместной работы будем использовать для этого мощь git'a? Надеюсь, аккаунты на github у нас есть?
Если нет, то нужно завести. Это не сложно. Ещё, очень советую установить интерфейс командной
строки для работы с github. Исчерпывающая и не сложная инструкция есть по этому адресу:
https://github.com/cli/cli/blob/trunk/docs/install\_linux.md

Антон: Да, принимается. Ждем.

Действие второе.
В своем кабинете Святослав, обхватив голову руками, погрузился в размышления.
Затем совершает следующие манипуляции:
1. Создается каталог проекта.
2. Создаются в каталоге три файла: instruction.txt, list.txt, README.md.
3. Выполняется команда git init, с помощью которой создается служебный каталог .git.
4. Созданные файлы наполняются содержимым. В файле README.md обычно помещается краткое (в одном-двух
предложениях) описание проекта. Файлы instruction.txt и list.txt представлены в листигах .
5. Командой git add *.txt .gitignore README.md файлы добавляются в индекс.
6. Командой git commit -m ``Проект документов'' создается первый коммит проекта.
7. Командой
 gh repo create instruction --public --gitignore TeX
	создается репозиторий на github.
8. 

Действие третье.
Кирилл работает в своем кабинете. Настроение прекрасное. Он только что изучил язык разметки
MarkDown. Ему не терпится что-то им разметить. 

Кирилл: Надо бы проверить содержимое почтового ящика. Вот! Отлично, что-то есть. 
Письмо от Святослава с какой-то ссылкой. Да, мы же договаривались. Очевидно, это проект
разрешительных документов. Сейчас посмотрим.
