В этой статье будет рассмотрена возможность использования git для совместной работы над 
некоторым проектом. Понятнее будет рассматривать эту работу на некотором вымышленном примере. 
Пример учебный и абсолютно вымышленный. Все аналогии и совпадении совершенно случайны.

Предполагается, что читатель уже знаком с основами работы с git. Здесь мы будем рассматривать,
в основном, работу в команде с помощью git.
Итак:
Действующие лица:
Антон - человек очень занятый, но добровольно взявший на себя бремя модерации проекта.
Кирилл - человек тоже занятый, но заинтересованный в успешности предприятия и добровольно
взявший на себя обязанности по личному участию в проекте.
Святослав - человек малозанятый и потому собирающийся погрузиться в проект с головой.
Некоторые разъяснения и комментарии будут вводиться от имени Автора.

Действие первое. 
Служебное помещение, где проходит очное совещание, в котором принимают участие все действующие лица.

Антон: Коллеги, есть интересная задача. Мы все так или иначе пользуемся нагревательными приборами
для удовлетворения в том числе и духовных потребностей своего организма. Но электронагревательный 
прибор, как известно является источником множества опасностей. Было бы хорошо разработать
документацию, необходимую для использования нашего чайника. Какие будут соображения?

Кирилл: Да все, в общем, известно. Нужно написать инструкцию по пользованию прибором и разрешение
на его использование. 
Затем их нужно будет подписать у соотвествующих отвественных работников и разместить все эти труды
рядом с используемым электронагревательным прибором. Затем уже все этим пользоваться. 

Святослав: Как я понимаю, предлагается мне возглавить эту работу? Я к этому готов. В скорости
пришлю на согласование проект предлагаемых документов. Думаю, будет правильно и эффективно
для совместной работы будем использовать для этого мощь git'a? Надеюсь, аккаунты на github у нас есть?
Если нет, то нужно завести. Это не сложно. Ещё, очень советую установить интерфейс командной
строки для работы с github. Исчерпывающая и не сложная инструкция есть по этому адресу:
https://github.com/cli/cli/blob/trunk/docs/install\_linux.md

Антон: Да, принимается. Ждем.

Действие второе.
В своем кабинете Святослав, обхватив голову руками, погрузился в размышления.
Святослав совершает следующие манипуляции:
1. Создает каталог проекта.
2. Создаёт в каталоге три файла: instruction.txt, permission.txt, README.md.
3. Выполняет команду git init, с помощью которой создается служебный каталог .git.
4. Созданные файлы наполняет содержимым. В файле README.md обычно помещается краткое (в одном-двух
предложениях) описание проекта. Файлы instruction.txt и permission.txt представлены в листигах .
5. Командой git add *.txt .gitignore README.md файлы добавляются в индекс.
6. Командой git commit -m ``Проект документов'' создается первый коммит проекта.
7. Командой
 gh repo create instruction --public --gitignore TeX
	создается репозиторий на github.

От автора:
 Команда gh --- это префикс команд интерфейса командной строки для управления
github. Прежде, чем управлять сервисом github нужно пройти авторизацию с помощью команды
  gh auth login

8. Командой 
	git remote add origin <address repository>
локальный git-репозиторий связывается с удаленным репозиторием по имени origin.
9. Теперь нужно файлы из локального репозитория поместить в удаленный репозиторий. Для этого
выполняется команда:
	git push --set-upstream origin master

От автора:
Здесь возникнут некоторые проблемы, связанные с тем, что при создании 
удаленного репозитория мы создали файл .gitignore. Фактически это означает что 
при сохранении локального репозитория в удаленный мы сливаем две ветки разработки.

10. Адрес удаленного репозитория, содержащего проект документации рассылается участникам
проекта. 

Действие третье.
Кирилл работает в своем кабинете. Настроение прекрасное. Он только что изучил язык разметки
MarkDown. Ему не терпится что-то им разметить. 

Кирилл: Надо бы проверить содержимое почтового ящика. Вот! Отлично, что-то есть. 
Письмо от Святослава с какой-то ссылкой. Да, мы же договаривались. Очевидно, это проект
разрешительных документов. Сейчас посмотрим.

От автора: Кирилл тоже пользуется интерфейсом командной строки для взаимодействия с сервером
github. Все манипуляции с сервером github можно делать с помощью web-интерфейса, нажимая
нужные кнопочки в браузере. Но в данной статье мы будем по возможности пользоваться интерфейсом
командно строки. 

Кирилл делает fork репозитория Святослава, одновременно
создавая локальную копию этого репозитория у себя на компьютере. Для этого выполняется следующая
команда:

gh repo fork svyatoslav/instuctions --clone

Затем он сразу же создает ветку для разработки под названием kirill-dev, переходит в  неё 

git branch -b kirill-dev

и добавляет в инструкцию еще несколько позиций (листинг ).
Делается фиксация этих изменений следующими командами

git add instruction.txt
git commit -m ``Kirill add prohibiting part''

Затем файл instruction.txt форматируется языком разметки MarkDown и результат сохраняется в
файле instruction.md.
Опять выполняется фиксация изменений проекта командами. 

git add instruction.md
git commit -m ``Kirill add markdown-file instruction.md''

Сохраняются изменения в удаленном репозитории. Это делается командой:

git push --set-upstream origin kirill-dev

Действие четвертое.
Некоторое время спустя Святослав на своем рабочем месте.

Святослав: Ага! Очевидно, Кирилл поработал над моим проектом. Мне пришел  Pull Request.
Сейчас разберемся. Ну что же, дополнения важные и правильные. Так он еще и разметил текст
инструкции MarkDown'ом? Очень правильно он сделал, что изменения записал в два разных commit'а.
В одном дополнения по тексту, а в другом разметка. Очень правильно.
От автора: Это действительно правильно вносить различные изменения в проект в разных commit'ах.
Какие-то можно внести в основную ветку, а какие-то оставить в другой ветке. Таким образом они
никуда не пропадут и всегда можно к ним будет вернуться. 
Святослав: Теперь нужно включить изменения предложенные Кириллом. Но я пока включу только те 
изменения, которые относятся к сути документа. А к форматированию вернусь позже.
От автора:
Так, как известно, что Антон не пользуется интернетом, создается копия локального 
репозитория в файле. Для этого используется команда

git bundle create instructions.bundle --all

Полученный в результате файл записывается на компакт-диск.

Действие пятое. 
В кабинете у Антона. Антон погружен в раздумья, заходит Святослав.
Святослав: Добрый день. Антон, я слышал ты с недоверием относишься к интернету, поэтому я принес
тебе твердую копию наших наработок. 
Антон: Да, по-моему, интернет --- это рассадник мерзостей. Хотя бы, на работе я стараюсь оградить себя от 
связанных с ним неприятностей. Кстати, к flash-накопителям я тоже отношусь с недоверием. Надеюсь, 
ты принес ваши наработки на настоящем православном CD-диске? 
Святослав: Да, конечно. Тем более, я научился, как правильно пользоваться
этим традиционным устройством, внимательнейшим образом изучив одну интересную статью.

Принесенный диск, вставляется в дисковод компьютера Антона. С помощью следующей команды разворачивается
локальная копия репозитория на компьютере антона

git clone instructions.bundle

Антон: Я посмотрю и что-нибудь подкорректирую. Загляну с оказией.
