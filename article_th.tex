\documentclass[14pt,a4paper]{article}
%\usepackage{theatre}
\usepackage[russian]{babel}
\usepackage[utf8]{inputenc}
\usepackage[T2A]{fontenc}
\usepackage{enumitem}
\usepackage{graphicx}

\usepackage{formatting} 

\begin{document}
\section{Введение}
В этой статье будет рассмотрена возможность использования git для совместной работы над 
некоторым проектом. Понятнее будет рассматривать эту работу на некотором вымышленном примере. 
Пример учебный и абсолютно вымышленный. Все аналогии и совпадении совершенно случайны.

Предполагается, что читатель уже знаком с основами работы с git. Здесь мы будем рассматривать,
в основном, работу в команде с помощью git.

\section{Собственно действие}
Итак:

Действующие лица:

Антон - человек очень занятый, при этом он является инициатором проекта.
Кирилл - человек тоже занятый, но заинтересованный в успешности предприятия и добровольно
взявший на себя обязанности по личному участию в проекте.
Святослав - человек малозанятый и потому собирающийся погрузиться в проект с головой.
Некоторые разъяснения и комментарии будут вводиться от имени Автора.

\action{Действие первое} 
\scene{Явление первое}
\remark{Служебное помещение, где проходит очное совещание, в котором принимают участие все действующие лица.}

\player{Антон} Коллеги, есть интересная задача. Мы все так или иначе пользуемся нагревательными приборами
для удовлетворения в том числе и духовных потребностей своего организма. Но электронагревательный 
прибор, как известно является источником множества опасностей. Было бы хорошо разработать
документацию, необходимую для использования нашего чайника. Какие будут соображения?

\player{Кирилл} Да все, в общем, известно. Нужно написать инструкцию по пользованию прибором и разрешение
на его использование. 
Затем их нужно будет подписать у соответствующих ответственных работников и разместить все эти труды
рядом с используемым электронагревательным прибором. Затем уже все этим пользоваться. 

\player{Святослав} Как я понимаю, предлагается мне возглавить эту работу? Я к этому готов. В скорости
пришлю на согласование проект предлагаемых документов. Думаю, будет правильно и эффективно
для совместной работы будем использовать для этого мощь git'a? Надеюсь, аккаунты на github у нас есть?
Если нет, то нужно завести. Это не сложно. Ещё, очень советую установить интерфейс командной
строки для работы с github. Исчерпывающая и не сложная инструкция есть по этому адресу:
https://github.com/cli/cli/blob/trunk/docs/install\_linux.md

\player{Кирилл} А зачем так сложно? Зачем на какой-то \texttt{git}? Разве нельзя попроще?

\player{Святослав} А как? Предложи.

\player{Кирилл} Ну вот, например. Один участник делает свою часть работы, то есть, создает какой-то файл.
Передает её другому. Другой что-то дополняет, передает третьему. Тот, тоже что-то делает. Затем,
встречаемся, обсуждаем, принимаем. Все, работа выполнена. Оценка --- <<отлично>>.

\player{Святослав} Да, а как обсуждать? Только при очных встречах? А как всем работать одновременно? Так, 
как ты предложил, каждый следующий может работать только после того, как свою часть закончил предыдущий.
Но самой главное, после нескольких итераций, у каждого участника будет какая-то своя версия и очень трудно
будет все совместить. Конечно, этот проект маленький, здесь можно и обойтися малой кровью и удовлетвориться
алгоритмом Кирилла. Но мы же должны расти, мы же должны быть готовыми работать над  крупными проектами?

\player{Антон} Да, а у меня ведь нет никаких ваших интернетов. Мне то как быть?

\player{Святослав} Слава создателю\footnote{Здесь имеется ввиду Линус Торвальдс --- создатель \texttt{git}}, в \texttt{git}'е есть технология \texttt{git bundle}. Можно обойтись 
и без доступа в сеть. 

\player{Антон} Ну хорошо, принимается. Ждем от Святослава предмета рассмотрения. 

\action{Действие второе.}
\scene{Явление первое.}
\remark{В своем кабинете Святослав, обхватив голову руками, пребывает в размышлениях.}
\player[Всплелснув руками]{Святослав}Да, а что я сижу? Я же обещал решительнейшим образом заняться
разработкой документации для чайника.
\remark{Склоняется над клавиатурой. Совершает следующие манипуляции:
	\begin{enumerate}
		\item Создает каталог проекта.
		\item Создаёт в каталоге три файла: instruction.txt, permission.txt, README.md.\sloppy
		\item Выполняет команду git init, с помощью которой создается служебный каталог .git.
		\item Созданные файлы наполняет содержимым. В файле README.md обычно помещается краткое (в одном-двух
предложениях) описание проекта. Файлы instruction.txt и permission.txt представлены в листигах 
\ref{list-instruction-first} и \ref{list-permission-first}.
\begin{figure}
	\centering
	\showtext{\input{instruction-first.txt}}{0.8}
	\caption{Первый вариант инструкции.}
	\label{list-instruction-first}
\end{figure}
\item Командой \examplecode{git add *.txt .gitignore README.md} файлы добавляются в индекс.
\item Командой \examplecode{git commit -m ``Проект документов''} создается первый коммит проекта.
\item Командой: \examplecode{gh repo create instruction --public --gitignore TeX}
	создается репозиторий на github.
	\end{enumerate}
\player{От автора}
{\normalfont Команда gh --- это префикс команд интерфейса командной строки для управления
github. Прежде, чем управлять сервисом github нужно пройти авторизацию с помощью команды
\examplecode{gh auth login}
  }

  \begin{enumerate}[resume]
	  \item  Командой: \examplecode{git remote add origin <address repository>}
локальный git-репозиторий связывается с удаленным репозиторием по имени origin.
\item  Теперь нужно файлы из локального репозитория поместить в удаленный репозиторий. Для этого
выполняется команда:
\examplecode{git push --set-upstream origin master}
\end{enumerate}

\player{От автора}
{\normalfont Здесь возникнут некоторые проблемы, связанные с тем, что при создании 
удаленного репозитория мы создали файл .gitignore. Фактически это означает что 
при сохранении локального репозитория в удаленный мы сливаем две ветки разработки.}

  \begin{enumerate}[resume]
\item Адрес удаленного репозитория, содержащего проект документации рассылается участникам
проекта. 
\end{enumerate}
%Результатом этих действий в рабочем каталоге, и удаленном репозитории появляются файлы, 
%представленые в листинге \ref{list-instruction-first} и листинге \ref{list-permission-first}.
}

\scene{Явление второе.}
\remark{Кирилл работает в своем кабинете. Настроение прекрасное. Он только что изучил язык разметки
MarkDown. Ему не терпится что-то им разметить. 
}
\player{Кирилл} Надо бы проверить содержимое почтового ящика. Вот! Отлично, что-то есть. 
Письмо от Святослава с какой-то ссылкой. Да, мы же договаривались. Очевидно, это проект
разрешительных документов. Сейчас посмотрим.

\player{От автора} Кирилл тоже пользуется интерфейсом командной строки для взаимодействия с сервером
github. Все манипуляции с сервером github можно делать с помощью web-интерфейса, нажимая
нужные кнопочки в браузере. Но в этом коллективе участники предпочитают пользоваться интерфейсом
командной строки. 

\remark{
Кирилл делает fork репозитория Святослава, одновременно
создавая локальную копию этого репозитория у себя на компьютере. Для этого выполняется следующая
команда:

\examplecode{gh repo fork svyatoslav/instuctions --clone}

Кстати, указав ключ \texttt{--fork-name} можно изменить имя \texttt{fork}-репозитория.

Затем он сразу же создает ветку для разработки под названием kirill-dev, переходит в  неё 

\examplecode{git branch -b kirill-dev}

и добавляет в инструкцию еще несколько позиций (листинг ).
Делается фиксация этих изменений следующими командами

\examplecode{git add instruction.txt
git commit -m ``Kirill add prohibiting part''
}
Следующим действием файл instruction.txt форматируется языком разметки MarkDown и результат 
сохраняется в файле instruction.md.
Опять выполняется фиксация изменений проекта командами. 

\examplecode{git add instruction.md
git commit -m ``Kirill add markdown-file instruction.md''
}
Сохраняются изменения в удаленном репозитории. Это делается командой:

\examplecode{git push --set-upstream origin kirill-dev}
}

\player{Антон}{Для создания запроса на добавление в репозиторий Святослава необходимо сделать так называемый
pull-request. Это такая специальная технология, которая позволяет предложить свои изменения 
в проект другого автора, обсудить его, и, если изменения принимаются, включить в проект.
Важно понимать, что  pull-request, это не функциональность git, это функциональность github.
Поэтому создать pull-reuest можно либо используя web-приложение github.com, либо 
используя интерфейс командной строки github --- gh. И в том и другом случае, при создании 
pull-request требуется указать:
\begin{itemize}
	\item Базовый репозиторий --- репозиторий, из которого создавался \texttt{fork};
	\item Ветка базового репозитория, обычно это \texttt{master};
	\item Репозиторий, из которого будут вливаться изменения, в данном случае это
	\item Ветка, в которой производились изменения и которые будут теперь вливаться в 
		базовый репозиторий, в данном случае это \texttt{kirill-dev}.
\end{itemize}
Создать \texttt{pull request} с помощью web-интерфейса не сложно. Для этого нужно:
\begin{itemize}
	\item в браузере зайти на сайт \texttt{github.com}, там войти в форкнутый репозиторий;
	\item выбрать в выпадающем списке разрабатываемую ветку;
	\item нажать большую зеленую кнопку \texttt{Compare \& Pull Request}.
\end{itemize}
С помощью интерфейса командно строки нужно набрать:
\examplecode{gh pr create}
Далее будет предложено указать те же позиции.
}

\scene{Явление третье.}
\remark{Некоторое время спустя Святослав на своем рабочем месте.}

\player{Святослав} Ага! Очевидно, Кирилл поработал над моим проектом. Мне пришел  Pull Request.
Сейчас разберемся. Ну что же, дополнения важные и правильные. Так он еще и разметил текст
инструкции MarkDown'ом? 
Теперь нужно принять изменения предложенные Кириллом и подготовить проект для Антона.

Для этого Святослав совершает следующие действия:
\begin{itemize}
	\item Сначала нужно посмотреть \texttt{Pull request'ы}, которые были получены.
		Это делается командой:
		\begin{verbatim}
		gh pr list
		\end{verbatim}
	\item Для того чтобы посмотреть содержимое выбранного по номеру \texttt{Pull request'а}
		выполняется следующая команда:
		\begin{verbatim}
		gh pr view 4
		\end{verbatim}
	\item Можно вступить в обсуждение, комментарий создается командой:
		\begin{verbatim}
		gh pr review 4 --comment --body ``Some text''
		\end{verbatim}
	\item В конце концов \texttt{Pull Request} принимается командой:
		\begin{verbatim}
		gh pr merge 4 
		\end{verbatim}
	\item Или отвергается и закрывается командой:
		\begin{verbatim}
		gh pr close 4 -c ``No!''
		\end{verbatim}
\end{itemize}
\player{От автора}
Так, как известно, что Антон не пользуется интернетом. Нужно создать копия локального 
репозитория в файле. 

Святослав выполняет следующую команду:

\begin{verbatim}
git bundle create instructions.bundle --all
\end{verbatim}

Полученный в результате файл записывается на компакт-диск.

\scene{Явление четвертое.}
\remark{Кирилл на своем рабочем месте. }

\player[за компьютером]{Кирилл} Так, отлично, вижу Святослав принял мои изменения. Обновлю fork его 
репозитория и свой локальный репозиторий, чтобы иметь у себя актуальную версию.

Кирилл выполняет следующие действия:
\begin{verbatim}
git fetch upstream			# Скачиваются изменения из 
									удаленного репозитория
git switch master			# Переключаемся на свою ветку master
git merge upstream/master	# Сливаем скачанные обновления 
								удаленной ветки с локальной веткой
git push					# Записываем сделанные изменения
								в свой удаленный репозиторий
\end{verbatim}

\action{Действие третье. }
\scene{Явление первое.}
\remark{В кабинете у Антона. Антон погружен в раздумья, заходит Святослав.}

\player{Святослав} Добрый день. Антон, я слышал ты с недоверием относишься к интернету, поэтому я принес
тебе твердую копию наших наработок. 

\player[отрываясь от раздумий]{Антон}: Да, по-моему, интернет --- это рассадник мерзостей. 
Хотя бы, на работе я стараюсь оградить себя от связанных с ним неприятностей. 
Кстати, к flash-накопителям я тоже отношусь с недоверием. Надеюсь, ты принес ваши наработки
на настоящем православном CD-диске? 

\player{Святослав} Да, конечно. Тем более, я научился, как правильно пользоваться
этим традиционным устройством, внимательнейшим образом изучив одну интересную статью.

\remark{
Антон вставляет принесенный Святославом диск в свой компьютер. С помощью следующей команды разворачивается
локальная копия репозитория на компьютере Антона
}

\examplecode{git clone instructions.bundle}

\player{От автора} Кстати, когда вы получаете создаете локальный репозиторий командой \texttt{git clone}, 
вы получает всю историю развития репозитория. Так, например, Антон может посмотреть что и когда 
сделал каждый из участников проекта. Для этого нужно сначала посмотреть историю проекта.
Делается это командой: \examplecode{git log --oneline} Результат её выполнения представлен
на рис.~\ref{pic-git-log}.
Важно знать, что \texttt{git log} очень мощная команда. С помощью неё можно получить массу информации
о всём процессе разработки проекта. Например, можно показать только те изменения, которые сделал 
определенный автор. Так коммиты, которые сделал Кирилл, можно посмотреть с помощью команды:
\examplecode{git log --oneline --author kirill}
Затем с помощью команды:
\examplecode{git diff 33dace7 f5248b6 -\textcompwordmark- instruction.txt}
можно посмотреть какие отличия между файлом \texttt{instruction.txt} во выбранных коммитах.
Отличия представляются в виде, показанном на рис.~\ref{pic-diff-result}.
Здесь в строках маркированных символами \texttt{@@}, определяется диапазон строк в сравниваемых
файлах, где найдены различия. Затем символом $-$ обозначается удаленная строка, а символом $+$
обозначается добавленная строка. 
\begin{figure}
	\centering
	\showtext{\input{git-log-1.txt}}{.8}
	\caption{Результат команды \texttt{git log}.}
	\label{pic-git-log}
\end{figure}

\begin{figure}
	\centering
	\showtext{\input{git-diff-1.txt}}{1}
	\caption{Результат команды \texttt{git diff}.}
	\label{pic-diff-result}
\end{figure}

\player{Антон} Я посмотрю и что-нибудь подкорректирую. Загляну с оказией. 
А как мне с помощью \texttt{git}'а правильно оформить результаты своей деятельности?
\player{Святослав} Правильно было бы создать новую ветку и все изменения проводить 
в новой ветке разработки. Новая ветка создается командой:
\examplecode{git checkout -b dev-anton}
Затем в рабочем каталоге проекта сделать все необходимы изменения, оформив их в одном
или нескольких коммитах. Можно сохранить в файле проект полностью. Для этого выполняется 
команда:
\examplecode{git bundle create <имя файла> --all}
или, что лучше в больших проектах записать в файл только свои коммиты (например, их 3):
\examplecode{git bundle create <имя файла> -3 dev-anton}
Кстати, это самый традиционный способо использования \texttt{git}, который и создавался
для такого распределенного использования. Всякие там \texttt{GitHub} и \texttt{GitLab}
были придуманы значительно позже.

\player[Всматриваясь в экран монитора]{Антон} Так, мои коллеги написали интересно, но это какой-то
фашизм\footnote{Идеология, однозначно осуждаемая всеми действующими лицами. Здесь употребляется в
фигуральном смысле}? Вся эта инструкция наводит мрак и ужас. Так и не захочется никакой чай пить.
Я добавлю небольшую преамбулу.

\begin{figure}
	\centering
	\showtext{\input{instruction_preambula.txt}}{.8}
	\caption{Преамбула к инструкции.}
	\label{listing-preambula}
\end{figure}


\remark{Антон добавляет в начало инструкции желаемый текст. По окончанию, он выполняет команду
	\examplecode{git diff}
	Результат этой команды представлен листинге~\ref{listing-preambula}, который показан, какие
	изменения внесены в проект, но еще не зафиксированы.
	Затем выполняются следующие действия:
	\examplecode{
		git add instruction.txt
	}
}

\scene{Явление второе.}
\remark{Святослав у себя на рабочем месте.}

\player[Включая компьютер.]{Святослав} Не буду ждать, пока Антон предложит свои изменения. Займусь
пока форматированием текста для печати. Как у нас принято, конечно, подготовлю документы к печати,
используя \LaTeX. Хорошо, что \texttt{git} позволяет работать надо проектом совместно. 

\remark{Используя файл instruction.txt, Святослав создает файл instruction.tex, который 
	представлен в листинге~\ref{listing-instruction-tex}.
}
 
\section{Заключение}

\end{document}
